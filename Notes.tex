\documentclass{report}

\input{preamble}
\input{macros}
\input{letterfonts}

\title{\Huge{Review over some Math} \\ \vspace{1cm} \normalsize{Notes on the book `All the Math You Missed But Need to Know for Graduate School' by Thomas A. Garrity}, \\ along with some personal thoughts and excercise solutions.}
\author{\huge{José Daniel Mejía C.}}
\date{2026}

\begin{document}

\maketitle
\newpage% or \cleardoublepage
% \pdfbookmark[<level>]{<title>}{<dest>}
\pdfbookmark[section]{\contentsname}{toc}
\tableofcontents
\pagebreak

\chapter{Summary over topics}
\section{Structure of mathematics}
\subsection{Equivalence problems}

\nt{The concept of two objects being the same depends on the field of study.}


\textbf{Equivalence} between two objects can be explained by the allowed Maps or functions between them.

\ex{}{A square and a circle are equivalent in topology; they can be stretched out from one to the other.
\\
They are not the same in differential geometry; one is smooth (differentiable) and the other is not.}

\textbf{Invariants} are properties that two or more object must satisfy if they are equivalent, however they themselves do not determine equivalence between them.

\ex{}{In topology, two circles are not equivalent to a single circle. The two circles have two connected components, whereas the one only has one connected component. Hence, both objects are not equivalent.
\\
However, a sphere and a circle have the same number of connected components, but they are not equivalent.}

\subsection{Functions}
Functions describe all kinds of situations, both in the real world and in an abstract matter.

Each field of study or area in mathematics focuses in a specific kind of function. Calculus studies differentiable functions fron $\mathbb{R} \to \mathbb{R}$, for example.

\pagebreak

\section{Topics}

\subsection{Linear Algebra}
Linear algebra studies \textbf{Linear Transformations} and \textbf{Vector Spaces}. A linear transformation has a matrix representation, given a basis for the vector space being considered.

Two or more different matrices can represent the same transformation, but over different choices of bases of the same vector space.

The key theorem in linear algebra explains several equivalent descriptions for when a matrix, and hence the associated transformation, is invertible.

Other important aspect is understanding the occurrence and usefulness of eigenvectors and eigenvalues.

\subsection{Real Analysis}
The definitions of \textbf{Limit}, \textbf{Continuity}, \textbf{Differentiation} and \textbf{Integration} are given in terms of $\epsilon$ and $\delta$. \textbf{Uniform Convergence} of functions is a key part of this topic.

\subsection{Differentiating Vector-Valued Functions}
The \textbf{Inverse Function Theorem} explains that a differentiable function $f : \mathbb{R} \to \mathbb{R}^n$ is invertible iff the determinant of its derivative (the Jacobian) is never zero.

It is important to understand what it means for a vector-valued function to be differentiable, and why its derivative must always be a linear map, represented by the Jacobian as a matrix.

Understanding the \textbf{Implicit function theorem} and how it relates to the Inverse Function Theorem is also relevant.

\subsection{Point Set Topology}
The concept of \textbf{Open Sets} and how a \textbf{Topology} can be built in terms of open sets must be understood. Also how a function is said to be \textbf{Continuous} in terms of open sets is key.

The standard topology in $\mathbb{R}^n$ and the \textbf{Heine-Borel Theorem} is covered, as well as what a metric space is and how a topology is defined using its metric.

\subsection{Classical Stokes' Theorem}
The calculus of vector fields covers the \textbf{Curl}, and \textbf{Divergence} of vector fields, the \textbf{Gradient} of a function and the \textbf{Path Integral} along a curve.

The \textbf{Divergence Theorem} and \textbf{Stokes' Theorem} are presented as classical extensions of the Fundamental Theorem of Calculus, it is relevant to understand why they are indeed extensions of this classic theorem.

\subsection{Differential Forms and Stokes' Theorem}
\textbf{Manifolds} are geometric objects. \textbf{Differential $k$-forms} allows calculus to be done on manifolds. There are several ways to define a manifold. How to think about and define differential $k$-forms, and how to take their \textbf{external derivative} is important.

Differential $k$-forms and external derivatives can be represented using vector fields, gradients, curls and divergences.

The statement of \textbf{Stokes' Theorem} in this context is covered, and special cases over the Divergence Theorem and Stokes' Theorem seen before.

\subsection{Curvature for Curves and Surfaces}
\textbf{Curvature} is a measure of the rate of change of the direction of tangent spaces of several geometric objects.

It is relevant to compute the curvature of curves on a plane, the curvature and torsion of a curve in space, and the two principal curvatures of a surface in space, in terms of the Hessian. 

\subsection{Geometry}
Different axioms build several geometries. For example, euclidean geometry assumes that given a line $l$ and a point $p$ not on $l$, there is exactly only one other line containing $p$ parallel to $l$. Elliptic and hyperbolic geometries assume different conditions.

Models for hyperbolic geometry, single elliptic geometry and double elliptic geometry are covered.

\subsection{Countability and the Axiom of Choice}
The concept of a \textbf{Countably Infinite} set is presented. For example, the Integers and Rationals are, while the Real numbers are not.

The statement of \textbf{The Axiom of Choice} and its (sometimes weird) equivalences should be covered.

\subsection{Elementary Number Theory}
An introduction to \textbf{Modular Arithmetic} is given. Importan explanations as why there are infinitely many primes, what is a \textbf{Diophintine Equation}, what is the \textbf{Euclidean Algorithm} and how it is linked to \textbf{Continuous Fractions} are covered.


\end{document}
